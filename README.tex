\documentclass[]{article}
\usepackage{lmodern}
\usepackage{amssymb,amsmath}
\usepackage{ifxetex,ifluatex}
\usepackage{fixltx2e} % provides \textsubscript
\ifnum 0\ifxetex 1\fi\ifluatex 1\fi=0 % if pdftex
  \usepackage[T1]{fontenc}
  \usepackage[utf8]{inputenc}
\else % if luatex or xelatex
  \ifxetex
    \usepackage{mathspec}
  \else
    \usepackage{fontspec}
  \fi
  \defaultfontfeatures{Ligatures=TeX,Scale=MatchLowercase}
\fi
% use upquote if available, for straight quotes in verbatim environments
\IfFileExists{upquote.sty}{\usepackage{upquote}}{}
% use microtype if available
\IfFileExists{microtype.sty}{%
\usepackage{microtype}
\UseMicrotypeSet[protrusion]{basicmath} % disable protrusion for tt fonts
}{}
\usepackage[margin=1in]{geometry}
\usepackage{hyperref}
\hypersetup{unicode=true,
            pdftitle={General Frailty Models: Shared, Joint and Nested Frailty Models with Prediction; Evaluation of Failure-Time Surrogate Endpoints},
            pdfauthor={Virginie Rondeau, Juan R. Gonzalez, Yassin Mazroui, Audrey Mauguen, Amadou Diakite, Alexandre Laurent, Myriam Lopez, Agnieszka Krol and Casimir L. Sofeu},
            pdfborder={0 0 0},
            breaklinks=true}
\urlstyle{same}  % don't use monospace font for urls
\usepackage{color}
\usepackage{fancyvrb}
\newcommand{\VerbBar}{|}
\newcommand{\VERB}{\Verb[commandchars=\\\{\}]}
\DefineVerbatimEnvironment{Highlighting}{Verbatim}{commandchars=\\\{\}}
% Add ',fontsize=\small' for more characters per line
\usepackage{framed}
\definecolor{shadecolor}{RGB}{248,248,248}
\newenvironment{Shaded}{\begin{snugshade}}{\end{snugshade}}
\newcommand{\KeywordTok}[1]{\textcolor[rgb]{0.13,0.29,0.53}{\textbf{#1}}}
\newcommand{\DataTypeTok}[1]{\textcolor[rgb]{0.13,0.29,0.53}{#1}}
\newcommand{\DecValTok}[1]{\textcolor[rgb]{0.00,0.00,0.81}{#1}}
\newcommand{\BaseNTok}[1]{\textcolor[rgb]{0.00,0.00,0.81}{#1}}
\newcommand{\FloatTok}[1]{\textcolor[rgb]{0.00,0.00,0.81}{#1}}
\newcommand{\ConstantTok}[1]{\textcolor[rgb]{0.00,0.00,0.00}{#1}}
\newcommand{\CharTok}[1]{\textcolor[rgb]{0.31,0.60,0.02}{#1}}
\newcommand{\SpecialCharTok}[1]{\textcolor[rgb]{0.00,0.00,0.00}{#1}}
\newcommand{\StringTok}[1]{\textcolor[rgb]{0.31,0.60,0.02}{#1}}
\newcommand{\VerbatimStringTok}[1]{\textcolor[rgb]{0.31,0.60,0.02}{#1}}
\newcommand{\SpecialStringTok}[1]{\textcolor[rgb]{0.31,0.60,0.02}{#1}}
\newcommand{\ImportTok}[1]{#1}
\newcommand{\CommentTok}[1]{\textcolor[rgb]{0.56,0.35,0.01}{\textit{#1}}}
\newcommand{\DocumentationTok}[1]{\textcolor[rgb]{0.56,0.35,0.01}{\textbf{\textit{#1}}}}
\newcommand{\AnnotationTok}[1]{\textcolor[rgb]{0.56,0.35,0.01}{\textbf{\textit{#1}}}}
\newcommand{\CommentVarTok}[1]{\textcolor[rgb]{0.56,0.35,0.01}{\textbf{\textit{#1}}}}
\newcommand{\OtherTok}[1]{\textcolor[rgb]{0.56,0.35,0.01}{#1}}
\newcommand{\FunctionTok}[1]{\textcolor[rgb]{0.00,0.00,0.00}{#1}}
\newcommand{\VariableTok}[1]{\textcolor[rgb]{0.00,0.00,0.00}{#1}}
\newcommand{\ControlFlowTok}[1]{\textcolor[rgb]{0.13,0.29,0.53}{\textbf{#1}}}
\newcommand{\OperatorTok}[1]{\textcolor[rgb]{0.81,0.36,0.00}{\textbf{#1}}}
\newcommand{\BuiltInTok}[1]{#1}
\newcommand{\ExtensionTok}[1]{#1}
\newcommand{\PreprocessorTok}[1]{\textcolor[rgb]{0.56,0.35,0.01}{\textit{#1}}}
\newcommand{\AttributeTok}[1]{\textcolor[rgb]{0.77,0.63,0.00}{#1}}
\newcommand{\RegionMarkerTok}[1]{#1}
\newcommand{\InformationTok}[1]{\textcolor[rgb]{0.56,0.35,0.01}{\textbf{\textit{#1}}}}
\newcommand{\WarningTok}[1]{\textcolor[rgb]{0.56,0.35,0.01}{\textbf{\textit{#1}}}}
\newcommand{\AlertTok}[1]{\textcolor[rgb]{0.94,0.16,0.16}{#1}}
\newcommand{\ErrorTok}[1]{\textcolor[rgb]{0.64,0.00,0.00}{\textbf{#1}}}
\newcommand{\NormalTok}[1]{#1}
\usepackage{graphicx,grffile}
\makeatletter
\def\maxwidth{\ifdim\Gin@nat@width>\linewidth\linewidth\else\Gin@nat@width\fi}
\def\maxheight{\ifdim\Gin@nat@height>\textheight\textheight\else\Gin@nat@height\fi}
\makeatother
% Scale images if necessary, so that they will not overflow the page
% margins by default, and it is still possible to overwrite the defaults
% using explicit options in \includegraphics[width, height, ...]{}
\setkeys{Gin}{width=\maxwidth,height=\maxheight,keepaspectratio}
\IfFileExists{parskip.sty}{%
\usepackage{parskip}
}{% else
\setlength{\parindent}{0pt}
\setlength{\parskip}{6pt plus 2pt minus 1pt}
}
\setlength{\emergencystretch}{3em}  % prevent overfull lines
\providecommand{\tightlist}{%
  \setlength{\itemsep}{0pt}\setlength{\parskip}{0pt}}
\setcounter{secnumdepth}{0}
% Redefines (sub)paragraphs to behave more like sections
\ifx\paragraph\undefined\else
\let\oldparagraph\paragraph
\renewcommand{\paragraph}[1]{\oldparagraph{#1}\mbox{}}
\fi
\ifx\subparagraph\undefined\else
\let\oldsubparagraph\subparagraph
\renewcommand{\subparagraph}[1]{\oldsubparagraph{#1}\mbox{}}
\fi

%%% Use protect on footnotes to avoid problems with footnotes in titles
\let\rmarkdownfootnote\footnote%
\def\footnote{\protect\rmarkdownfootnote}

%%% Change title format to be more compact
\usepackage{titling}

% Create subtitle command for use in maketitle
\newcommand{\subtitle}[1]{
  \posttitle{
    \begin{center}\large#1\end{center}
    }
}

\setlength{\droptitle}{-2em}

  \title{General Frailty Models: Shared, Joint and Nested Frailty Models with
Prediction; Evaluation of Failure-Time Surrogate Endpoints}
    \pretitle{\vspace{\droptitle}\centering\huge}
  \posttitle{\par}
    \author{Virginie Rondeau, Juan R. Gonzalez, Yassin Mazroui, Audrey Mauguen,
Amadou Diakite, Alexandre Laurent, Myriam Lopez, Agnieszka Krol and
Casimir L. Sofeu}
    \preauthor{\centering\large\emph}
  \postauthor{\par}
      \predate{\centering\large\emph}
  \postdate{\par}
    \date{10 décembre 2018}


\begin{document}
\maketitle

\section{Description:}\label{description}

The following several classes of frailty models using a penalized
likelihood estimation on the hazard function but also a parametric
estimation can be fit using this R package:

\begin{verbatim}
1) A shared frailty model (with gamma or log-normal frailty distribution) and Cox proportional hazard model. Clustered and recurrent survival times can be studied.
2) Additive frailty models for proportional hazard models with two correlated random effects (intercept random effect with random slope).
3) Nested frailty models for hierarchically clustered data (with 2 levels of clustering) by including two iid gamma random effects.
4) Joint frailty models in the context of the joint modelling for recurrent events with terminal event for clustered data or not. A joint frailty model for two semi-competing risks and clustered data is also proposed.
5) Joint general frailty models in the context of the joint modelling for recurrent events with terminal event data with two independent frailty terms.
6) Joint Nested frailty models in the context of the joint modelling for recurrent events with terminal event, for hierarchically clustered data (with two levels of clustering) by including two iid gamma random effects.
7) Multivariate joint frailty models for two types of recurrent events and a terminal event.
8) Joint models for longitudinal data and a terminal event.
9) Trivariate joint models for longitudinal data, recurrent events and a terminal event. 
10) Joint frailty models for the validation of surrogate endpoints in multiple randomized clinical trials with failure-time endpoints 
    
\end{verbatim}

Prediction values are available (for a terminal event or for a new
recurrent event). Left-truncated (not for Joint model), right-censored
data, interval-censored data (only for Cox proportional hazard and
shared frailty model) and strata are allowed. In each model, the random
effects have the gamma or normal distribution. Now, you can also
consider time-varying covariates effects in Cox, shared and joint
frailty models (1-5). The package includes concordance measures for Cox
proportional hazards models and for shared frailty models.

\subsection{Installation}\label{installation}

The R package frailtypack is available on cran at
\url{https://cran.r-project.org/web/packages/frailtypack/index.html}

You can install frailtypack from github with:

\begin{Shaded}
\begin{Highlighting}[]
\CommentTok{# install.packages("frailtypack")}
\NormalTok{devtools}\OperatorTok{::}\KeywordTok{install_github}\NormalTok{(}\StringTok{"socale/frailtypack"}\NormalTok{)}
\end{Highlighting}
\end{Shaded}

\subsection{Example}\label{example}

This is a basic example which shows you how use frailtypack in the
context of the validation of a candidate surrogate endpoint. We use
genarated data.

\begin{Shaded}
\begin{Highlighting}[]
\NormalTok{## Data Generation}
\NormalTok{data.sim <-}\StringTok{ }\KeywordTok{jointSurrSimul}\NormalTok{(}\DataTypeTok{n.obs=}\DecValTok{600}\NormalTok{, }\DataTypeTok{n.trial =} \DecValTok{30}\NormalTok{,}\DataTypeTok{cens.adm=}\FloatTok{549.24}\NormalTok{, }
    \DataTypeTok{alpha =} \FloatTok{1.5}\NormalTok{, }\DataTypeTok{theta =} \FloatTok{3.5}\NormalTok{, }\DataTypeTok{gamma =} \FloatTok{2.5}\NormalTok{, }\DataTypeTok{zeta =} \DecValTok{1}\NormalTok{, }\DataTypeTok{sigma.s =} \FloatTok{0.7}\NormalTok{, }
    \DataTypeTok{sigma.t =} \FloatTok{0.7}\NormalTok{, }\DataTypeTok{rsqrt =} \FloatTok{0.8}\NormalTok{, }\DataTypeTok{betas =} \OperatorTok{-}\FloatTok{1.25}\NormalTok{, }\DataTypeTok{betat =} \OperatorTok{-}\FloatTok{1.25}\NormalTok{, }\DataTypeTok{full.data =} \DecValTok{0}\NormalTok{, }
    \DataTypeTok{random.generator =} \DecValTok{1}\NormalTok{, }\DataTypeTok{seed =} \DecValTok{0}\NormalTok{, }\DataTypeTok{nb.reject.data =} \DecValTok{0}\NormalTok{)}
\end{Highlighting}
\end{Shaded}

\begin{Shaded}
\begin{Highlighting}[]
\CommentTok{# Estimation}
\NormalTok{joint.surro.sim.MCGH <-}\StringTok{ }\KeywordTok{jointSurroPenal}\NormalTok{(}\DataTypeTok{data =}\NormalTok{ data.sim, }\DataTypeTok{int.method =} \DecValTok{2}\NormalTok{,}
    \DataTypeTok{nb.mc =} \DecValTok{300}\NormalTok{, }\DataTypeTok{nb.gh =} \DecValTok{20}\NormalTok{)}
\end{Highlighting}
\end{Shaded}

\begin{Shaded}
\begin{Highlighting}[]
\CommentTok{# Result}
\KeywordTok{summary}\NormalTok{(joint.surro.sim.MCGH)}
\end{Highlighting}
\end{Shaded}


\end{document}
